\documentclass[a4paper,12pt]{article}
\usepackage{libertinus}
%----------------------------------------------------------------------------------------
%	PACKAGES
%----------------------------------------------------------------------------------------
\usepackage{url}
\usepackage{parskip} 	

%other packages for formatting
\RequirePackage{color}
\RequirePackage{graphicx}
\usepackage[usenames,dvipsnames]{xcolor}
\usepackage[a4paper,top=1cm,bottom=1cm,left=1cm,right=1cm]{geometry}

%tabularx environment
\usepackage{tabularx}

%for lists within experience section
\usepackage{enumitem}

% centered version of 'X' col. type
\newcolumntype{C}{>{\centering\arraybackslash}X} 

%to prevent spillover of tabular into next pages
\usepackage{supertabular}
\usepackage{tabularx}
\newlength{\fullcollw}
\setlength{\fullcollw}{0.47\textwidth}

%custom \section
\usepackage{titlesec}				
\usepackage{multicol}
\usepackage{multirow}

%CV Sections inspired by: 
%http://stefano.italians.nl/archives/26
\titleformat{\section}{\Large\scshape\raggedright}{}{0em}{}[\titlerule]
\titlespacing{\section}{0pt}{10pt}{10pt}

%for publications
\usepackage{biblatex}

%Setup hyperref package, and colours for links
\usepackage[unicode, draft=false]{hyperref}
\definecolor{linkcolour}{rgb}{0,0.2,0.6}
\hypersetup{colorlinks,breaklinks,urlcolor=linkcolour,linkcolor=linkcolour}
\addbibresource{citations.bib}
\setlength\bibitemsep{1em}

%for social icons
\usepackage{fontawesome5}

% job listing environments
\newenvironment{jobshort}[2]
    {
    \begin{tabularx}{\linewidth}{@{}l X r@{}}
    \textbf{#1} & \hfill &  #2 \\[3.75pt]
    \end{tabularx}
    }
    {
    }

\newenvironment{joblong}[2]
{
	\begin{tabularx}{\linewidth}{@{}l X r@{}}
		\textbf{#1} & \hfill & #2 \\[2pt]
	\end{tabularx}
	\begin{itemize}[nosep,leftmargin=1em,itemsep=2pt,label=--]
	}
	{
	\end{itemize}
}


%----------------------------------------------------------------------------------------
%	BEGIN DOCUMENT
%----------------------------------------------------------------------------------------
\begin{document}

% non-numbered pages
\pagestyle{empty} 

%----------------------------------------------------------------------------------------
%	TITLE
%----------------------------------------------------------------------------------------
	\begin{tabularx}{\linewidth}{@{} C @{}}
	\Huge{Paulin Aubert} \\[7.5pt]
	\href{https://github.com/apolain/}{\raisebox{-0.05\height}\faGithub\ Apolain} \ $|$ \ 
	\href{https://www.linkedin.com/in/paulinaubert/}{\raisebox{-0.05\height}\faLinkedin\ Paulin Aubert} \ $|$ \ 
	\href{https://apolain.github.io/}{\raisebox{-0.05\height}\faGlobe \ https://apolain.github.io/} \ $|$ \ 
	\href{mailto:paulinaubert@orange.fr}{\raisebox{-0.05\height}\faEnvelope \ paulinaubert@orange.fr} \\
%	\href{tel:+xxxxxxx}{\raisebox{-0.05\height}\faMobile \ +xxxxxxx} \\
	\end{tabularx}

%----------------------------------------------------------------------------------------
% EXPERIENCE SECTIONS
%----------------------------------------------------------------------------------------
	\section{Summary}
	\textbf{Research Scientist} with a \textbf{PhD in Applied Mathematics}, specialized in reinforcement learning, stochastic control, numerical optimization, and deep learning.
	Experienced in designing theoretically grounded learning algorithms and deploying them in real-world, large-scale systems.
	
	%Experience
	\section{Work Experience}
	
	\begin{joblong}{Quantitative Researcher \& Analyst — Exiom Partners (Jun. 2021 - Present)}{}
		\item[$\bullet$] \textbf{Quantitative Analyst (Full-Time, Tier-1 global bank — CCR/XVA)} \hfill Sept. 2025 -- Present
		\begin{itemize}	
			\item Contribute to the development and maintenance of a quantitative library for derivative pricing, XVA computation, and regulatory risk metrics (C++, Python).
			\item Contribute to the design and implementation of a Python-based tool automating performance and stability monitoring across the CCR/XVA library. \\
		\end{itemize}
		
		\item[$\bullet$] \textbf{Doctoral Researcher (Industrial PhD, CIFRE)} \hfill Jun. 2022 -- Dec. 2025
		\begin{itemize}
			\item Industry-sponsored doctoral research conducted under joint supervision with Université Paris-Saclay and Laboratoire de Mathématiques et Modélisation d'Évry (LaMME).
			\item Conduct research on reinforcement learning and learning-based numerical methods for high-dimensional stochastic control problems.
			\item Design theoretically grounded algorithms combining dynamic programming, deep neural networks, and Monte Carlo methods.
			\item Study convergence, stability, and generalization properties of learning-based control algorithms.
			\item Apply developed methods to optimal stopping, sequential decision-making, and market making problems.
			\item Publish research papers and present results at international conferences.
		\end{itemize}
		
		\item[$\bullet$] \textbf{Quantitative Analyst (Part-Time, Concurrent with PhD)} \hfill Jun. 2022 -- Sept. 2025
		\begin{itemize}
			\item \textbf{Large French retail bank (ALM team):} Led the redesign of the existing C++/C\# library into a flexible, object-oriented Python architecture for pricing and risk applications. \\
			Designed the object-oriented Python architecture of the new framework and supervised the technical work of the development team throughout the project. 
			Conducted model reviews and developments covering swap, bond, and swaption pricing, yield-curve stripping, and short-rate model calibration.
			\item \textbf{Multiple financial institutions (credit risk projects):} 
			Delivered various credit risk projects, including the design of credit-scoring models, the analysis of non-performing loan portfolios, and the review of provisioning methodologies. \\
			Performed statistical analysis and developed data-driven models in Python, applying techniques from statistics, data analysis, and machine learning to assess credit quality and model default risk. \\
		\end{itemize}
		
		\item[$\bullet$] \textbf{Quantitative Analyst (Full-Time)} \hfill Nov. 2021 -- Jun. 2022
		\begin{itemize}
			\item Contribute to the development and maintenance of an internal Python quantitative finance library.
			\item Participate in various quantitative projects for financial institutions.
			\item Conduct exploratory research on learning-based methods for stochastic control, leading to the definition and writing of the doctoral research proposal. \\
		\end{itemize}
%	\end{joblong}
%	
%	\begin{joblong}{}{}
		\item[$\bullet$] \textbf{Internal Tools \& IT Infrastructure} \hfill Nov. 2021 -- Present
		\begin{itemize}
			\item Led the design, development, and maintenance of internal IT tools, including a Django-based planning and recruitment platform (Python, HTML, CSS, JavaScript, Azure).
			\item Administered and maintained a self-hosted GitLab environment to support version control, CI/CD workflows, and team collaboration. \\
		\end{itemize}
		
		\item[$\bullet$] \textbf{Quantitative Research Intern} \hfill May 2021 -- Nov 2021
		\begin{itemize}
			\item Conducted research on Default Risk Charge (DRC) requirements under the Fundamental Review of the Trading Book (FRTB) framework.
			\item Developed a multi-period Merton model for credit risk analysis and performed theoretical and numerical studies of dependence structures using copula models.
		\end{itemize}
	\end{joblong}

%----------------------------------------------------------------------------------------
%	EDUCATION
%----------------------------------------------------------------------------------------
	\section{Education}
	\begin{tabularx}{\linewidth}{@{}l X@{}}	
		2022 -- 2025 & \textbf{PhD in Applied Mathematics} — Université Paris-Saclay, Laboratoire de Mathématiques et de Modélisation d'Évry (LaMME), Exiom Partners \\
		& \textbf{Thesis:} \textit{Learning-based numerical methods for stochastic control in finance.} \\
		2019 -- 2021 & \textbf{Master’s Degree in Quantitative Finance} — Université Paris-Saclay \\
		& Graduated with honors. \\
		2016 -- 2019 & \textbf{Double bachelor’s degree in Economics and Mathematics} — Le Mans Université \\
		& Graduated with honors.
	\end{tabularx}

%----------------------------------------------------------------------------------------
%	PUBLICATIONS
%----------------------------------------------------------------------------------------
	\section{Publications}
	\begin{refsection}
	\nocite{*}
	\printbibliography[heading=none]
	\end{refsection}
	
%----------------------------------------------------------------------------------------
%	INTERESTS
%----------------------------------------------------------------------------------------
	
	\section{Research Interests}
	\begin{itemize}[nosep]
		\item Reinforcement Learning and Sequential Decision-Making
		\item Learning-based Numerical Methods
		\item Stochastic Optimization and Control
		\item Deep Learning for High-Dimensional Problems
		\item Theory-guided Machine Learning
	\end{itemize}
	

%----------------------------------------------------------------------------------------
%	SKILLS
%----------------------------------------------------------------------------------------
	\section{Skills}
	\begin{tabularx}{\linewidth}{@{}l X@{}}
		\textbf{Machine Learning} & Reinforcement learning, supervised and unsupervised learning, deep neural networks. \\
		\textbf{Data Science} & Data preprocessing, exploratory data analysis, statistical modeling, and empirical evaluation of machine learning models. \\
		\textbf{Optimization \& Control} & Stochastic control, dynamic programming, optimal stopping, policy optimization. \\
		\textbf{Numerical Methods} & Monte Carlo methods, high-dimensional approximation, PDE-based and learning-based solvers. \\
		\textbf{Programming} & Python (PyTorch, NumPy, SciPy, Pandas), C++. Experience with large-scale experiments and research codebases. \\
		\textbf{Research} & Algorithm design, theoretical analysis, experimental validation, academic writing and peer-reviewed publications. \\
		\textbf{Languages} & French (native), English (fluent).
	\end{tabularx}

\end{document}
