\documentclass[a4paper,11pt]{article}
\usepackage{libertinus}
%----------------------------------------------------------------------------------------
%	PACKAGES
%----------------------------------------------------------------------------------------
\usepackage{url}
\usepackage{parskip} 	

%other packages for formatting
\RequirePackage{color}
\RequirePackage{graphicx}
\usepackage[usenames,dvipsnames]{xcolor}
\usepackage[a4paper,top=1cm,bottom=0.5cm,left=1cm,right=1cm]{geometry}

%tabularx environment
\usepackage{tabularx}

%for lists within experience section
\usepackage{enumitem}

% centered version of 'X' col. type
\newcolumntype{C}{>{\centering\arraybackslash}X} 

%to prevent spillover of tabular into next pages
\usepackage{supertabular}
\usepackage{tabularx}
\newlength{\fullcollw}
\setlength{\fullcollw}{0.47\textwidth}

%custom \section
\usepackage{titlesec}				
\usepackage{multicol}
\usepackage{multirow}

%CV Sections inspired by: 
%http://stefano.italians.nl/archives/26
\titleformat{\section}{\Large\scshape\raggedright}{}{0em}{}[\titlerule]
\titlespacing{\section}{0pt}{10pt}{10pt}

%for publications
\usepackage{biblatex}

%Setup hyperref package, and colours for links
\usepackage[unicode, draft=false]{hyperref}
\definecolor{linkcolour}{rgb}{0,0.2,0.6}
\hypersetup{colorlinks,breaklinks,urlcolor=linkcolour,linkcolor=linkcolour}
\addbibresource{citations.bib}
\setlength\bibitemsep{1em}

%for social icons
\usepackage{fontawesome5}

% job listing environments
\newenvironment{jobshort}[2]
    {
    \begin{tabularx}{\linewidth}{@{}l X r@{}}
    \textbf{#1} & \hfill &  #2 \\[3.75pt]
    \end{tabularx}
    }
    {
    }

\newenvironment{joblong}[2]
    {
    \begin{tabularx}{\linewidth}{@{}l X r@{}}
    \textbf{#1} & \hfill &  #2 \\[3.75pt]
    \end{tabularx}
    \begin{minipage}[t]{\linewidth}
    \begin{itemize}[nosep,after=\strut, leftmargin=1em, itemsep=3pt,label=--]
    }
    {
    \end{itemize}
    \end{minipage}    
    }

%----------------------------------------------------------------------------------------
%	BEGIN DOCUMENT
%----------------------------------------------------------------------------------------
\begin{document}

% non-numbered pages
\pagestyle{empty} 

%----------------------------------------------------------------------------------------
%	TITLE
%----------------------------------------------------------------------------------------
	\begin{tabularx}{\linewidth}{@{} C @{}}
	\Huge{Paulin Aubert} \\[7.5pt]
	\href{https://github.com/apolain/}{\raisebox{-0.05\height}\faGithub\ Apolain} \ $|$ \ 
	\href{https://www.linkedin.com/in/paulinaubert/}{\raisebox{-0.05\height}\faLinkedin\ Paulin Aubert} \ $|$ \ 
	\href{https://apolain.github.io/}{\raisebox{-0.05\height}\faGlobe \ https://apolain.github.io/} \ $|$ \ 
	\href{mailto:paulinaubert@orange.fr}{\raisebox{-0.05\height}\faEnvelope \ paulinaubert@orange.fr} \\
%	\href{tel:+xxxxxxx}{\raisebox{-0.05\height}\faMobile \ +xxxxxxx} \\
	\end{tabularx}

%----------------------------------------------------------------------------------------
% EXPERIENCE SECTIONS
%----------------------------------------------------------------------------------------
	\section{Summary}
	Quantitative Analyst and PhD candidate in Applied Mathematics (defense Dec. 2025), specializing in stochastic control, numerical methods, and machine learning applied to finance.
	Experienced in developing and implementing quantitative models through a CIFRE PhD bridging academic research and industry, with strong expertise in modeling, simulation, and risk management.
	
	%Experience
	\section{Work Experience}

	\begin{joblong}{Quantitative Consultant \& PhD Candidate -- Exiom Partners}{Nov 2021 - Present}
	\item PhD researcher focusing on numerical methods for stochastic control, leveraging machine learning and reinforcement learning techniques. Actively involved in writing research papers and presenting results at international conferences.
	\item Working as a part-time Quant in industry, contributing to several projects in risk and model development:
		\begin{itemize}
			\item \textbf{Tier-1 global bank (CCR/XVA team):} Contributing to the development of the pricing library and the automation of monitoring tests. Implementations in Python and C++.
			\item \textbf{Large French retail bank (team):} Led a C++/C\# library rationalisation project within the ALM team. Designed a new Python library and conducted model reviews and developments covering swap, bond, and swaption pricing, yield-curve stripping, and short-rate model calibration.
			\item \textbf{Multiple financial institutions (credit risk projects):} Delivered credit-risk projects including the design of credit-scoring grids and the review of provisioning methodologies for non-performing loans.
		\end{itemize}
	\item Led the development and maintenance of internal IT tools, including a Django-based planning and recruitment platform and a self-hosted GitLab environment.
	\end{joblong}

	\begin{jobshort}{Quand intern -- Exiom Partners}{May 2021 - Nov 2021}
	Conducted research on Default Risk Charge (DRC) requirements under the Fundamental Review of the Trading Book (FRTB) framework.
	Developed a multi-period Merton model for credit risk analysis and performed theoretical and numerical studies of dependence structures using copula models.
	\end{jobshort}

%----------------------------------------------------------------------------------------
%	EDUCATION
%----------------------------------------------------------------------------------------
	\section{Education}
	\begin{tabularx}{\linewidth}{@{}l X@{}}	
		2022 -- 2025 & \textbf{PhD in Applied Mathematics} — Université Paris-Saclay \\
		& Thesis: Learning-based numerical methods for stochastic control in finance. \\
		2019 -- 2021 & \textbf{Master’s Degree in Quantitative Finance} — Université Paris-Saclay
	\end{tabularx}

%----------------------------------------------------------------------------------------
%	PUBLICATIONS
%----------------------------------------------------------------------------------------
	\section{Publications}
	\begin{refsection}
	\nocite{*}
	\printbibliography[heading=none]
	\end{refsection}

%----------------------------------------------------------------------------------------
%	SKILLS
%----------------------------------------------------------------------------------------
	\section{Skills}
	\begin{tabularx}{\linewidth}{@{}l X@{}}
		\textbf{Quantitative Modeling} & Stochastic control, optimal stopping, market making, credit and counterparty risk (FRTB–DRC, XVA). \\
		\textbf{Numerical Methods} & Monte Carlo simulation, PDE schemes, function approximation, optimization and calibration techniques. \\
		\textbf{Programming} & Python (NumPy, Pandas, SciPy, PyTorch) and C++. Software engineering with Git, CI/CD, unit testing, and Linux environments. \\
		\textbf{Scientific Rigor} & Strong analytical and methodological skills developed through academic research, including paper writing, technical presentations, and structured problem-solving. \\
		\textbf{Languages} & French (native), English (fluent). 
	\end{tabularx}

\end{document}
