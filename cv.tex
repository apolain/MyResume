\documentclass[a4paper,12pt]{article}
\usepackage{libertinus}
%----------------------------------------------------------------------------------------
%	PACKAGES
%----------------------------------------------------------------------------------------
\usepackage{url}
\usepackage{parskip} 	

%other packages for formatting
\RequirePackage{color}
\RequirePackage{graphicx}
\usepackage[usenames,dvipsnames]{xcolor}
\usepackage[a4paper,top=1cm,bottom=1cm,left=1cm,right=1cm]{geometry}

%tabularx environment
\usepackage{tabularx}

%for lists within experience section
\usepackage{enumitem}

% centered version of 'X' col. type
\newcolumntype{C}{>{\centering\arraybackslash}X} 

%to prevent spillover of tabular into next pages
\usepackage{supertabular}
\usepackage{tabularx}
\newlength{\fullcollw}
\setlength{\fullcollw}{0.47\textwidth}

%custom \section
\usepackage{titlesec}				
\usepackage{multicol}
\usepackage{multirow}

%CV Sections inspired by: 
%http://stefano.italians.nl/archives/26
\titleformat{\section}{\Large\scshape\raggedright}{}{0em}{}[\titlerule]
\titlespacing{\section}{0pt}{10pt}{10pt}

%for publications
\usepackage{biblatex}

%Setup hyperref package, and colours for links
\usepackage[unicode, draft=false]{hyperref}
\definecolor{linkcolour}{rgb}{0,0.2,0.6}
\hypersetup{colorlinks,breaklinks,urlcolor=linkcolour,linkcolor=linkcolour}
\addbibresource{citations.bib}
\setlength\bibitemsep{1em}

%for social icons
\usepackage{fontawesome5}

% job listing environments
\newenvironment{jobshort}[2]
    {
    \begin{tabularx}{\linewidth}{@{}l X r@{}}
    \textbf{#1} & \hfill &  #2 \\[3.75pt]
    \end{tabularx}
    }
    {
    }

\newenvironment{joblong}[2]
    {
    \begin{tabularx}{\linewidth}{@{}l X r@{}}
    \textbf{#1} & \hfill &  #2 \\[3.75pt]
    \end{tabularx}
    \begin{minipage}[t]{\linewidth}
    \begin{itemize}[nosep,after=\strut, leftmargin=1em, itemsep=3pt,label=--]
    }
    {
    \end{itemize}
    \end{minipage}    
    }

%----------------------------------------------------------------------------------------
%	BEGIN DOCUMENT
%----------------------------------------------------------------------------------------
\begin{document}

% non-numbered pages
\pagestyle{empty} 

%----------------------------------------------------------------------------------------
%	TITLE
%----------------------------------------------------------------------------------------
	\begin{tabularx}{\linewidth}{@{} C @{}}
	\Huge{Paulin Aubert} \\[7.5pt]
	\href{https://github.com/apolain/}{\raisebox{-0.05\height}\faGithub\ Apolain} \ $|$ \ 
	\href{https://www.linkedin.com/in/paulinaubert/}{\raisebox{-0.05\height}\faLinkedin\ Paulin Aubert} \ $|$ \ 
	\href{https://apolain.github.io/}{\raisebox{-0.05\height}\faGlobe \ https://apolain.github.io/} \ $|$ \ 
	\href{mailto:paulinaubert@orange.fr}{\raisebox{-0.05\height}\faEnvelope \ paulinaubert@orange.fr} \\
%	\href{tel:+xxxxxxx}{\raisebox{-0.05\height}\faMobile \ +xxxxxxx} \\
	\end{tabularx}

%----------------------------------------------------------------------------------------
% EXPERIENCE SECTIONS
%----------------------------------------------------------------------------------------
	\section{Summary}
	\textbf{Quantitative Analyst} and \textbf{PhD candidate in Applied Mathematics} (defense Dec. 2025), specializing in stochastic control, numerical methods, and machine learning applied to finance.
	Experienced in developing and implementing quantitative models through a CIFRE PhD bridging academic research and industry.
	
	%Experience
	\section{Work Experience}

	\begin{joblong}{Quantitative Analyst \& PhD Candidate -- Exiom Partners}{Nov 2021 - Present}
	\item PhD researcher (Laboratoire de Mathématiques et Modélisation d'Evry (LaMME)):
		\begin{itemize}
			\item[$\bullet$] Conducts research on numerical and learning-based methods for stochastic control, with applications to pricing, optimal stopping, and market making.
			\item[$\bullet$] Develops and analyzes algorithms that are both theoretically grounded and computationally efficient, combining stochastic control theory, machine learning and reinforcement learning techniques.
			\item[$\bullet$] Supervised research internships on deep hedging and on state-of-the-art reinforcement learning approaches for optimal stopping problems.
			\item[$\bullet$] Co-authors research papers and presents results at international conferences. \\
		\end{itemize}
		
	\item Working as a part-time Quant in industry, contributing to several projects in risk and model development:
		\begin{itemize}
			\item[$\bullet$] \textbf{Tier-1 global bank (CCR/XVA team):} 
			Quantitative analyst contributing to the development and maintenance of the quantitative library used for derivative pricing, XVA computation, and regulatory risk metrics, primarily implemented in C++ and Python. \\
			Participated in a Python-based project to automate the monitoring of pricing library performance, contributing to both methodological design and implementation aspects. \\
			\item[$\bullet$] \textbf{Large French retail bank (ALM team):} Led the redesign of the existing C++/C\# library into a flexible, object-oriented Python architecture for pricing and risk applications. \\
			Designed the object-oriented Python architecture of the new framework and supervised the technical work of the development team throughout the project. 
			Conducted model reviews and developments covering swap, bond, and swaption pricing, yield-curve stripping, and short-rate model calibration. \\
			\item[$\bullet$] \textbf{Multiple financial institutions (credit risk projects):} 
			Delivered various credit risk projects, including the design of credit-scoring models, the analysis of non-performing loan portfolios, and the review of provisioning methodologies. \\
			Performed statistical analysis and developed data-driven models in Python, applying techniques from statistics, data analysis, and machine learning to assess credit quality and model default risk. \\
		\end{itemize}
	\item Internal tools development and IT infrastructure:
		\begin{itemize}
			\item[$\bullet$] Led the design, development, and maintenance of internal IT tools, including a Django-based planning and recruitment platform (Python, HTML, CSS, JavaScript, Azure).
			\item[$\bullet$] Administered and maintained a self-hosted GitLab environment to support version control, CI/CD workflows, and team collaboration.
		\end{itemize}
	\end{joblong}

	\begin{jobshort}{Quant intern -- Exiom Partners}{May 2021 - Nov 2021}
	Conducted research on Default Risk Charge (DRC) requirements under the Fundamental Review of the Trading Book (FRTB) framework.
	Developed a multi-period Merton model for credit risk analysis and performed theoretical and numerical studies of dependence structures using copula models.
	\end{jobshort}

%----------------------------------------------------------------------------------------
%	EDUCATION
%----------------------------------------------------------------------------------------
	\section{Education}
	\begin{tabularx}{\linewidth}{@{}l X@{}}	
		2022 -- 2025 & \textbf{PhD in Applied Mathematics} — Université Paris-Saclay, Laboratoire de Mathématiques et de Modélisation d'Evry (LaMME), Exiom Partners \\
		& \textbf{Thesis:} \textit{Learning-based numerical methods for stochastic control in finance.} \\
		2019 -- 2021 & \textbf{Master’s Degree in Quantitative Finance} — Université Paris-Saclay \\
		& Graduated with honors. \\
		2016 -- 2019 & \textbf{Double bachelor’s degree in Economics and Mathematics} — Le Mans Université \\
		& Graduated with honors.
	\end{tabularx}

%----------------------------------------------------------------------------------------
%	PUBLICATIONS
%----------------------------------------------------------------------------------------
	\section{Publications}
	\begin{refsection}
	\nocite{*}
	\printbibliography[heading=none]
	\end{refsection}

%----------------------------------------------------------------------------------------
%	SKILLS
%----------------------------------------------------------------------------------------
	\section{Skills}
	\begin{tabularx}{\linewidth}{@{}l X@{}}
		\textbf{Financial Modeling} & Market making and optimal liquidation with price impact, order book dynamics, stochastic order flow modeling, and pricing models in quantitative finance. \\
		\textbf{Numerical Methods} & Monte Carlo simulation, PDE-based methods, and learning-based approaches (machine and reinforcement learning) for numerical and stochastic control problems. \\
		\textbf{Data Science} & Data preprocessing, statistical analysis, and machine learning with experience in model design, training, and evaluation using neural networks. \\
		\textbf{Programming} & Python (NumPy, Pandas, SciPy, PyTorch, Dash, Django) and C++. Experience with Git and Linux environments. \\
		\textbf{Software Engineering} & Version control, CI/CD workflows, unit testing, and deployment automation. \\
		\textbf{Scientific Rigor} & Analytical mindset and methodological precision. Experience in academic writing, technical presentation, and structured problem-solving. \\
		\textbf{Languages} & French (native), English (fluent).
	\end{tabularx}

\end{document}
